\section{Conclusion}
\label{sec:conclusions}

Our results demonstrate that for learning tasks on graph-based structures, the inductive bias introduced by GNNs offers a significant benefit over other modeling choices.
For the specific task of type prediction, we show that GNNs perform better than state of the art modeling choices like bi-directional RNNs.
We also confirm through experiments that the right choice of edges in the graph has a significant effect on the performance of the learning task.
Despite these results, we have not yet validated one of our initial assumptions, that a graph neural net is actually approximating some underlying probabilistic graphical model.
We believe a larger research question on formalizing GNNs through the lens of probabilistic graphical models, and proving (or disproving) their correspondence is an important open question.

\section{Task split-up}
\begin{itemize}[]
	\item \textbf{Alex.} Graph neural network implementation, analysis, report writing.
	\item \textbf{Katie.} Graph neural network implementation, analysis, report writing.
	\item \textbf{Shashank.} Typescript parsing and graph generation, analysis, report writing.
\end{itemize}
