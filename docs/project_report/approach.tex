\section{Approach}
\label{sec:approach}

To solve the approximate type inference problem, we take a strong inductive bias over the types of predictors that we would expect to work on programs.
Specifically, we take advantage of our knowledge of the important local and nonlocal interactions between various parts of programs, similar to how convolutional networks assume some strong relationship between local pixels~\cite{henaff2015deep}.
There are three specific priors we bake into our solution:

\paragraph{Nodes near a variable in the AST reflect something about that variable's type.}
For instance, consider the following code snippet:
\begin{lstlisting}
  let x = y * 2;
\end{lstlisting}
A human trying to infer the type of the variable \texttt{x} in the above code may decide that since the result of multiplying something by a number is almost always a number, \texttt{x} is probably a number.
This corresponds exactly to two types of edges that we include in our generated graph, an \texttt{AstChild} edge from parents to their child nodes in the AST, and an \texttt{AstParent} edge from children to their parent in the AST.

\paragraph{Nodes near a variable in the file reflect something about that variable's type.}
Consider the following code snippet:
\begin{lstlisting}
  let x = 1;
  let y = x;
  ...
  let x = "foo";
\end{lstlisting}
It is clear that both ordering and locality matter in determining the type of \texttt{y}: in the AST, the two statements assigning to \texttt{x} are equidistant and directionally indistinguishable, but the actual \emph{token stream} of file reflects that the first assignment to \texttt{x} precedes the assignment to \texttt{y}, meaning that it may be more likely to affect the type of \texttt{y}, and is also closer, meaning the two statements are more likely to be related.
Because of this, we include two more types of edges in the induced graph, a \texttt{TokenNext} edge from a token to its neighbor in the source file, and a \texttt{TokenPrev} edge which is the reverse.

\paragraph{Nodes that use or define the same variable give information about each other.}
Consider the following:
\begin{lstlisting}
  let x = 1;
  ...
  let y = a + (b + (... + x));
\end{lstlisting}
Although the two statements may be arbitrarily nonlocal in both the AST and the source file, they are clearly related, in that \texttt{x}'s type probably influences \texttt{y}'s type.
Because of this, we include two final types of edges, a \texttt{UseDef} edge from a variable's definition to its use, and a \texttt{UseUse} edge linking any two variable usages.

%%% Local Variables:
%%% TeX-master: "main"
%%% End: